\section{The Versioning Ontology}

The model underlying The Versioning Ontology (VersOn) utilizes three primary concepts: version, attribute, and change.
The model always relates at least two versions which represent the data object being versioned.
An attribute is the part of a version which changes, causing a new version.
A change represents the difference between two versions through their attributes, formed by the presence, absence, or alteration in a version.
In order to represent all three kinds of change, addition, invalidation, and modification, the concepts are arranged and related in three different formations based on the presence or absence of attributes.

\subsection{Modification}

The modify relation occurs when an attribute appears in both versions and the attributes' value are different.
In Figure \ref{ModificationFig}, a modification is captured between two versions.
\begin{figure}
	\centering
	\vspace{0.0in} % normally the command here would be \includegraphics
	%	\includegraphics{figures/Addition.png}
	\begin{tikzpicture}[every node/.style={draw, rectangle}]
	\begin{scope}[node distance=20mm and 20mm]
	\node (c) [scale=1.25] at (1,0) {Change M};
	\node (1) [above left=of c, scale=1.25] {Version 1};
	\node (2) [above right=of c, scale=1.25] {Version 2};
	\node (a1) [below =of 1, scale=1.25] {Attribute 1};
	\node (a2) [below =of 2, scale=1.25] {Attribute 2};
	
	\draw [line width=2pt,->] (a1) -- (c);
	\draw [line width=2pt,->] (c) -- (a2);
	\draw [line width=2pt, ->] (1) -- (a1);
	\draw [line width=2pt, ->] (2) -- (a2);
	\end{scope}
	\end{tikzpicture}
	\caption{Model of the relationships between Versions 1 and 2 when modifying Attribute 1 from Version 1 as a result of Change M, resulting in Attribute 2 from Version 2.}
	\label{ModificationFig}  % the \label command comes AFTER the caption
\end{figure}
Each version has an attribute, Attribute 1 and Attribute 2, respectively.
A modification concept, a subclass of the change concept, connects the two attributes, denoting that the values described by the attribute differ.
The connectivity between versions is maintained through the implied relationship across attributes.
The attributes are directly associated with the respective version.

The specific values pertaining to Attribute 1 and Attribute 2 are not captured by VersOn because acknowledging that a difference exists is more important.
Extending the model to properly communicate the significance of a modification for a wide variety of domains would require sizable domain knowledge and would be outside of a data versioning scientist's domain.
In addition, the model would essentially begin storing a copy of the data set, leading to space and redundancy concerns.

In some applications, a modification is represented as an invalidation followed by an addition.
The representation has a couple of problems.
The first is that the sequence of changes implies that there is an intermediary stage where all the modified values have been invalidated but not added.
The intermediary stage constitutes a new version of the data object, even though only one change is being made.
The second issue with using the two change sequence is that the two states of attribute, before and after the modification, becomes disassociated.
Any new attribute being added to the version is not necessarily associated with an attribute in the left-handed version.
Establishing the association between attributes after an invalidation-addition sequence is the same as and more concisely expressed as a single modification.

\subsection{Addition}

In Figure \ref{AdditionFig}, the addition construction differs from the modify construction by the absence of Attribute 1.
\begin{figure}[b]
	\centering
	\vspace{0.0in} % normally the command here would be \includegraphics
	%	\includegraphics{figures/Addition.png}
	\begin{tikzpicture}[every node/.style={draw, rectangle}]
	\begin{scope}[node distance=20mm and 20mm]
	\node (c) [scale=1.25] at (1,0) {Change A};
	\node (1) [above left=of c, scale=1.25] {Version 1};
	\node (2) [above right=of c, scale=1.25] {Version 2};
	\node (a) [below =of 2, scale=1.25] {Attribute 2};
	
	\draw [line width=2pt,->] (1) -- (c);
	\draw [line width=2pt,->] (c) -- (a);
	\draw [line width=2pt, ->] (2) -- (a);
	\end{scope}
	\end{tikzpicture}
	\caption{Model of the relationships between Versions 1 and 2 when adding an Attribute 2 to Version 2 as a result of Change A.}
	\label{AdditionFig}  % the \label command comes AFTER the caption
\end{figure}
The absence creates a disconnect between Version 1 and Change A.
In order to maintain version connectivity between versions, Version 1 must be reconnected to the other concepts in the model.
A property is used to create a path from Version 1 to Change A, maintaining the connectivity between Version 1 and Version 2.
The path does not show that Version 1 informs or creates Attribute 2, while the case may be true.
The construction was also chosen to create a symmetric orientation with the invalidation change.


\subsection{Invalidation}

The invalidation construction has a missing attribute on the right-hand side of the relation, contrary to the addition construction.
As a result of the invalidation, an attribute no longer exists in the right-hand version.
As seen in Figure \ref{InvalidationFig}, the invalidation change concept matches to the Version 2 object.
\begin{figure}
	\centering
	\vspace{0.0in} % normally the command here would be \includegraphics
	%	\includegraphics{figures/Addition.png}
	\begin{tikzpicture}[every node/.style={draw, rectangle}]
	\begin{scope}[node distance=15mm and 20mm]
	\node (c) [scale=1.25] at (1,0) {Change I};
	\node (1) [above left=of c, scale=1.25] {Version 1};
	\node (2) [above right=of c, scale=1.25] {Version 2};
	\node (a) [below =of 1, scale=1.25] {Attribute 1};
	
	\draw [line width=2pt,->] (a) -- (c);
	\draw [line width=2pt,->] (c) -- (2);
	\draw [line width=2pt, ->] (1) -- (a);
	\end{scope}
	\end{tikzpicture}
	\caption{Model of the relationships between Versions 1 and 2 when invalidating Attribute 1 from Version 1 as a result of Change I.}
	\label{InvalidationFig}  % the \label command comes AFTER the caption
\end{figure}
Just like in the addition model, the invalidation construction maintains a link between the two version objects in order to maintain version continuity.
In the invalidation case, it makes more conceptual sense, however, because Version 2 invalidates Attribute 1 by omitting the attribute.

