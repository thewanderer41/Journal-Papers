\section{Conclusion}

Dot-decimal identifiers are labels often used to indicate the categorical amount of change a new version introduces to a data set.
The change distance computed using VersOn provides a more regular, precise method to evaluate change using standardized semantics by separating change into types and enumerating individual changes.
The counts for GCMD Keywords were acquired by querying each transition's versioning graph.
The change distance did not reveal an order of magnitude separation between major and minor versions, but suggested a threshold indicating a full new release.
The change metric also showed that GCMD Keywords is a data set which generally adds data with each version release.
The analysis of Version 8.5 highlights the dynamic between data producer and consumer roles.
Using URI best-practice or GCMD Keywords consumer practice, the number of changes amounted to a new full keyword release.
Because the data producer is the authority on version labeling, a minor version number was assigned based on different metrics.
The VersOn based change distance enables consumer contextualized assessments based on utilization.