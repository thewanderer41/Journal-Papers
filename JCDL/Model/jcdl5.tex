\section{Change Log Analysis}

With a trade-off of 14 HTML lines for every visible line and 40 HTML code lines to each visible line, space utilization is a very present concern.
Table \ref{table:Cu_changelog_table1} shows the size of each encoding of the change log as well as the percentage in size as compared to either of the files involved in the version transition.
\begin{table}
	\caption{Copper change log size: 1st transition.}
	\label{table:Cu_changelog_table1}
	\centering
	\begin{tabular}{@{}rrrr@{}}
		\toprule
		Encoding Type & File Size (Bytes) & \% of File 1 & \% of File 2 \\
		\midrule
		Text&	140131&	41.3152&	59.9580\\
		RDFa&	2032823&	599.343&	869.787\\
		Turtle&	1538772&	453.680&	658.396\\
		JSON-LD&	3500067&	1031.93&	1497.57\\
		\bottomrule
	\end{tabular}
\end{table}
`Text' denotes the encoding control where no structured data is included into the change log.
The RDFa encoding is almost 6 times the size of the original files.
The encoded change log exceeds the size of the control by more than ten fold, meaning over 90\% reduction in change log performance.
A separate file was generated in turtle format to observe whether taking just the linked-data values would reduce the information to a more manageable size, but the turtle file was still over 4 times the size of the original files.
Adopting the versioning model and encoding it into a change log will very likely require significant storage investment.

Another way to evaluate the performance of the change log is to look at the number of change entries compared to the number of changed values in the Copper database's case spreadsheet cells.
To determine the number of cells affected by a change, the number of cells added by new rows is summed with the number of cells added by new columns, using the width and length of Version 2.
The cells affected by removals is based on the length and width of Version 1.
The number of remaining equivalent values between the two files is 23940.
Since modifications are reported cell-by-cell, the number of cells affected is equal to the number of modifications, 2628.
The rows and columns that modification affects are not available because the changes appear inconsistently across the rows and columns meaning a reported value would be misleading.
The complete counts are reported in Table \ref{table:Cu_cells}.
\begin{table}
	\caption{Changes to Copper Data.}
	\label{table:Cu_cells}
	\centering
	\begin{tabular}{@{}rrrr@{}}
		\toprule
		Change Type&	Rows&	Columns&	Cells Affected\\ \midrule
		Add&	1&	16&	10995\\
		Invalidate&	21&	2&	2145\\
		Modify&NA&NA& 2628\\
		\bottomrule
	\end{tabular}
\end{table}

The triples used to explain changes as a percentage of the cells affected are reported in Table \ref{table:Cu_change}.
\begin{table}
	\caption{Change capture compression in Copper Data.}
	\label{table:Cu_change}
	\centering
	\begin{tabular}{@{}rrr@{}}
		\toprule
		Change Type&	Triples&	\% of Cells Affected\\ \midrule
		Add&	17&	0.065\%\\
		Invalidate&	23&	1.1\%\\
		Modify&	2628&	100\%\\
		\bottomrule
	\end{tabular}
\end{table}
Smaller percentages indicate how well one triple can explain changes to multiple cells by compressing the number of entries.
Notice that additions are much more efficient in explaining changes than invalidations due to a skew towards column additions.
Invalidations explained changes to rows primarily while additions mostly explained changes to columns, but since columns are much longer, additions ended up scoring higher on efficiency.
Modification triples do not compress change information well and also account for more than a majority of the changes to the data, meaning that modification triples most likely account for the bloat in the physical representation of the triples.
Not represented in the change log are the unmodified cells which account for 89.02\% of the matching cells between the Copper files.
The analysis indicates that while addition and invalidation may be very efficient in expressing changes, improvements to encoding and modification capture are needed to bring down the storage costs of automated change logs.
